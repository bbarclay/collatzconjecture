\section{Detailed Proofs}

\subsection{Cryptographic Framework Proofs}

\begin{proof}[Detailed proof of Theorem \ref{thm:one_way}]
The one-way property follows from the exponential growth of the predecessor space:
\begin{enumerate}
\item For any odd $n$, consider the equation $\frac{3m + 1}{2^k} = n$
\item This implies $3m + 1 = n2^k$ for some $k \geq 1$
\item Therefore $m = \frac{n2^k - 1}{3}$ must be an integer
\item For each $k$, this gives at most one valid predecessor
\item The number of required $k$ values grows with $n$
\item Thus finding a predecessor requires checking $O(\log n)$ possibilities
\end{enumerate}
\end{proof}

\begin{proof}[Detailed proof of Theorem \ref{thm:avalanche}]
The avalanche effect emerges from the carry propagation:
\begin{enumerate}
\item A change in bit $i$ affects bit $i+1$ through multiplication by 3
\item The addition of 1 creates a carry chain
\item Each carry propagates upward with probability $\frac{1}{2}$
\item After $k$ steps, approximately $k/2$ bits are affected
\item This matches the ideal avalanche criterion asymptotically
\end{enumerate}
\end{proof}

\subsection{Measure Theory Proofs}

\begin{proof}[Detailed proof of Theorem \ref{thm:tau_dist}]
The distribution of $\tau$ follows from:
\begin{enumerate}
\item For $\tau(n) = k$, we need $3n + 1 \equiv 0 \pmod{2^k}$
\item This gives $n \equiv -\frac{1}{3} \pmod{2^k}$
\item The solution exists uniquely in each residue class
\item Therefore $P(\tau = k) = 2^{-k}$ asymptotically
\item The error term comes from boundary effects
\end{enumerate}
\end{proof}

\begin{proof}[Detailed proof of Theorem \ref{thm:measure_preserve}]
Measure preservation follows from:
\begin{enumerate}
\item For any arithmetic progression $P(a,d)$
\item The preimage $T^{-1}(P(a,d))$ is a union of progressions
\item The total density equals the original density
\item This extends to the generated $\sigma$-algebra
\end{enumerate}
\end{proof}

\subsection{Information Theory Proofs}

\begin{proof}[Detailed proof of Theorem \ref{thm:entropy}]
The entropy reduction follows from:
\begin{enumerate}
\item Initial entropy increase is $\log_2(3)$ bits
\item Division by $2^{\tau(n)}$ reduces entropy by $\tau(n)$ bits
\item Net change is $\log_2(3) - \tau(n)$ bits
\item Expected value is negative by $\tau$ distribution
\end{enumerate}
\end{proof}

\subsection{Global Behavior Proofs}

\begin{proof}[Detailed proof of Theorem \ref{thm:cycle_prevent}]
Cycle prevention follows from:
\begin{enumerate}
\item Any cycle must contain both odd and even numbers
\item Even numbers strictly decrease until reaching an odd
\item Odd steps have controlled growth by entropy bounds
\item Large $\tau$ events force eventual descent
\end{enumerate}
\end{proof}

\begin{proof}[Detailed proof of Theorem \ref{thm:global_descent}]
Global descent follows from:
\begin{enumerate}
\item Large $\tau$ events occur with positive probability
\item Each such event reduces the value significantly
\item The ergodic theorem ensures infinitely many occurrences
\item This prevents unbounded growth almost surely
\end{enumerate}
\end{proof} 