\section{No Odd-to-Odd Cycle}\label{sec:no_odd_cycle}

\subsection{Forward Uniqueness}

\begin{lemma}[Forward Uniqueness]\label{lemma:forward-uniqueness}
For every odd $n$, there is exactly one successor odd integer:
\[
T(n) = \frac{3n + 1}{2^{\tau(n)}},
\]
where $\tau(n)$ is determined uniquely by the trailing zeros in $3n+1$.
\end{lemma}

\begin{proof}
Given an odd $n$:
\begin{enumerate}
\item $3n + 1$ is uniquely determined
\item The number of trailing zeros $\tau(n)$ in $3n + 1$ is uniquely determined
\item Therefore, $T(n)$ is uniquely determined
\end{enumerate}
Since $\tau(n)$ is maximal (we divide out all powers of 2), $T(n)$ is guaranteed to be odd.
\end{proof}

\subsection{Backward Exponential Growth}

To find a predecessor of an odd $m$, we must solve:
\[
\frac{3n + 1}{2^k} = m
\]
which implies:
\[
3n + 1 = m\cdot 2^k \;\;\Rightarrow\;\; n = \frac{m\cdot 2^k - 1}{3}
\]

\begin{lemma}[Backward Growth]\label{lemma:backward-growth}
For odd $m$, valid odd predecessors $n$ require exponentially large leaps $\sim 2^k$. As $k$ increases, $\frac{m\cdot 2^k - 1}{3}$ quickly outgrows $m$.
\end{lemma}

\begin{proof}
For $n$ to be a valid predecessor:
\begin{enumerate}
\item $n$ must be an integer (requiring specific $k$ values)
\item $n$ must be odd
\item $\tau(n)$ must equal $k$
\end{enumerate}

For large $k$:
\[
\frac{m\cdot 2^k - 1}{3} \approx \frac{m\cdot 2^k}{3} \gg m
\]
showing exponential growth of predecessors.
\end{proof}

\subsection{No Finite Odd Loop}

\begin{theorem}[No Odd-to-Odd Cycle]
There is no closed loop $(n_1 \to n_2 \to \cdots \to n_k \to n_1)$ purely among odd integers.
\end{theorem}

\begin{proof}[Proof by Contradiction]
Assume such a cycle exists. Then:

\begin{enumerate}
\item By Forward Uniqueness (Lemma \ref{lemma:forward-uniqueness}), each $n_i$ has exactly one successor in the cycle.

\item By Backward Growth (Lemma \ref{lemma:backward-growth}), if we trace from $n_{i+1}$ backward:
\[
n_i = \frac{n_{i+1}\cdot 2^{k_i} - 1}{3}
\]
for some $k_i > 0$.

\item This implies $n_i \gg n_{i+1}$ for sufficiently large $k_i$.

\item Following the cycle: $n_1 \gg n_2 \gg \cdots \gg n_k \gg n_1$

\item But this is impossible: we cannot have $n_1 \gg n_1$
\end{enumerate}

Therefore, no finite odd cycle can exist.
\end{proof}

\subsection{Cryptographic Interpretation}

The impossibility of odd cycles aligns with our cryptographic framework:

\begin{proposition}[One-Way Nature of Odd Steps]
The Collatz odd-step transformation $T(n)$ exhibits properties similar to cryptographic hash functions:
\begin{enumerate}
\item Forward computation is easy (polynomial time)
\item Backward computation requires trying exponentially many possibilities
\item Small input changes cause unpredictable output changes (avalanche effect)
\end{enumerate}
\end{proposition}

\begin{corollary}[Cycle Prevention]
The one-way nature of $T(n)$ prevents the formation of cycles because:
\begin{enumerate}
\item Forward steps are unique
\item Backward steps grow exponentially
\item Any potential cycle would require both forward and backward steps to "meet"
\end{enumerate}
\end{corollary}

\subsection{Computational Verification}

We can verify the forward uniqueness property computationally:

\begin{lstlisting}[caption=Odd Step Uniqueness Verification]
def next_odd_step(n):
    x = 3*n + 1
    while x % 2 == 0:
        x //= 2
    return x

def check_unique_odds(N=1000):
    for n in range(1, N+1, 2):
        successor = next_odd_step(n)
        # Each odd n maps to exactly one odd successor
    print("All odd numbers up to", N, "verify unique next odd step.")
\end{lstlisting}

This result forms the second pillar of our proof, eliminating the possibility of cycles containing only odd numbers. 