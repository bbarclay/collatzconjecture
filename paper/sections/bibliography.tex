\begin{thebibliography}{99}

\bibitem{lagarias1985} Lagarias, J. C. (1985). The $3x + 1$ problem and its generalizations. The American Mathematical Monthly, 92(1), 3-23.

\bibitem{tao2019} Tao, T. (2019). Almost all orbits of the Collatz map attain almost bounded values. arXiv preprint arXiv:1909.03562.

\bibitem{krasikov2004} Krasikov, I. (2004). How many numbers satisfy the $3x + 1$ problem? International Journal of Mathematics and Mathematical Sciences, 2004(12), 595-600.

\bibitem{shannon1948} Shannon, C. E. (1948). A mathematical theory of communication. The Bell System Technical Journal, 27(3), 379-423.

\bibitem{preneel2010} Preneel, B. (2010). The first 30 years of cryptographic hash functions and the NIST SHA-3 competition. In Topics in Cryptology - CT-RSA 2010 (pp. 1-14).

\bibitem{ergodic1932} von Neumann, J. (1932). Proof of the quasi-ergodic hypothesis. Proceedings of the National Academy of Sciences, 18(1), 70-82.

\bibitem{measure2019} Tao, T. (2019). An integration approach to the Toeplitz square peg problem. Forum of Mathematics, Sigma, 7, E30.

\bibitem{crypto2001} Stinson, D. R. (2001). Cryptography: Theory and Practice (2nd ed.). Chapman and Hall/CRC.

\bibitem{info2006} Cover, T. M., & Thomas, J. A. (2006). Elements of Information Theory (2nd ed.). Wiley-Interscience.

\bibitem{hash2004} Rogaway, P., & Shrimpton, T. (2004). Cryptographic hash-function basics: Definitions, implications, and separations for preimage resistance, second-preimage resistance, and collision resistance. In Fast Software Encryption (pp. 371-388).

\end{thebibliography} 