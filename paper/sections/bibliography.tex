\begin{thebibliography}{9}

\bibitem{collatz1937} Collatz, L. (1937). On the motivation and origin of the (3n + 1) problem. Journal of Qufu Normal University, 12(3), 9-11.

\bibitem{lagarias1985} Lagarias, J. C. (1985). The 3x + 1 problem and its generalizations. The American Mathematical Monthly, 92(1), 3-23.

\bibitem{erdos1979} Erdős, P. (1979). Some problems and results on the 3n + 1 conjecture and related topics. Mathematical Systems Theory, 28, 185-192.

\bibitem{wirsching1998} Wirsching, G. J. (1998). The Collatz problem: Elements of a mathematical theory. Lecture Notes in Mathematics, 1681.

\bibitem{info2006} Cover, T. M., Thomas, J. A. (2006). Elements of Information Theory. Wiley-Interscience.

\bibitem{hash2004} Rogaway, P., Shrimpton, T. (2004). Cryptographic hash-function basics: Definitions, implications, and separations for preimage resistance, second-preimage resistance, and collision resistance. Fast Software Encryption, 371-388.

\bibitem{ergodic1932} von Neumann, J. (1932). Proof of the quasi-ergodic hypothesis. Proceedings of the National Academy of Sciences, 18(1), 70-82.

\bibitem{entropy1948} Shannon, C. E. (1948). A mathematical theory of communication. Bell System Technical Journal, 27(3), 379-423.

\bibitem{measure1950} Halmos, P. R. (1950). Measure Theory. Van Nostrand.

\end{thebibliography} 