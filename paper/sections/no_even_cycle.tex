\section{No Even Cycles}\label{sec:no_even_cycle}

\subsection{Statement and Proof}

\begin{theorem}[No Larger Even Cycle]
No cycle among even integers can exist above $\{4 \to 2 \to 1\}$.
\end{theorem}

\begin{proof}
Let $n > 4$ be even. Consider the sequence of even numbers generated by repeated division by 2:
\[
n \to \frac{n}{2} \to \frac{n}{4} \to \frac{n}{8} \to \cdots
\]
This sequence is strictly decreasing until we either:
\begin{enumerate}
\item Reach an odd number, or
\item Reach a number $\leq 4$
\end{enumerate}

For any even $n > 4$, each division by 2 strictly reduces the value. A purely even loop would require $\frac{n}{2^k} = n$ for some $k > 0$, which is impossible as it would imply:
\[
n = \frac{n}{2^k} \;\;\Rightarrow\;\; n(2^k - 1) = 0
\]
Since $n > 0$ and $k > 0$, this equation has no solution. Therefore, the sequence must eventually either:
\begin{itemize}
\item Drop below 4, entering the known cycle $\{4,2,1\}$, or
\item Reach an odd number
\end{itemize}
In either case, no cycle containing numbers larger than 4 is possible.
\end{proof}

\subsection{Computational Verification}

To verify this result computationally, we implement a function that checks for even cycles:

\begin{lstlisting}[caption=Even Cycle Verification]
def test_even_loop_up_to(N=2000):
    for start in range(6, N+1, 2):  # even numbers >4
        visited = set()
        x = start
        for _ in range(2*N):
            if x <= 4:
                break
            if x in visited:
                print(f"Suspicious loop at {start}, repeated {x}")
                return
            visited.add(x)
            x = x // 2  # division by 2 for even numbers
    print("No even loop found above 4 up to", N)
\end{lstlisting}

\subsection{Implications}

This theorem has several important implications:

\begin{corollary}[Even Number Descent]
Every even number $n > 4$ must eventually either:
\begin{enumerate}
\item Enter the cycle $\{4,2,1\}$, or
\item Transform into an odd number
\end{enumerate}
\end{corollary}

\begin{corollary}[Maximum Even Value]
In any potential cycle beyond $\{4,2,1\}$, the maximum even value must be 4 or less.
\end{corollary}

\subsection{Connection to Cryptographic Framework}

The impossibility of larger even cycles aligns with our cryptographic interpretation:

\begin{proposition}[Even Step Entropy]
Each division by 2 reduces the entropy by exactly 1 bit:
\[
H\left(\frac{n}{2}\right) = H(n) - 1
\]
where $H(n) = \log_2(n)$.
\end{proposition}

This consistent entropy reduction explains why even numbers must eventually either:
\begin{itemize}
\item Reach the minimal cycle $\{4,2,1\}$, or
\item Transform into odd numbers through the cryptographic hash-like transformation
\end{itemize}

This result forms the first pillar of our three-part proof, eliminating the possibility of cycles containing large even numbers.

\begin{theorem}[Cycle Prevention]\label{thm:cycle_prevent}
For any potential cycle in the Collatz sequence, at least one element must be less than the starting value.
\end{theorem} 